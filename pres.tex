\documentclass{beamer}

\usepackage[labelformat=empty]{caption}

\mode<presentation>
{
  \usetheme{Berkeley}
  \usecolortheme{seagull}
  \setbeamercovered{transparent}
}

\title{Nuclear Engineering is a thing}
\author{Jeffrey Seifried}
\institute{Ad Delivery Team, Yelp}
\date{Advanced Learning Group, \texttt{2015-03-16}}

\AtBeginSubsection[]
{
  \begin{frame}<beamer>{Outline}
    \tableofcontents[currentsection,currentsubsection]
  \end{frame}
}


\begin{document}

\begin{frame}
  \titlepage
\end{frame}

\begin{frame}{Outline}
  \tableofcontents
\end{frame}


\section{About Me}

    \begin{frame}{I've studied Nuclear Engineering for over a decade}{... and all I got was this stupid tshirt}

        \begin{itemize}

            \item BS at University of Maryland, College Park
            \begin{itemize}
                \item Became a nuclear reactor operator
                \item Spent 4 summers + 1 semester \\ at the U.S. Nuclear Regulatory Commission
            \end{itemize}

            \pause

            \item PhD at UC, Berkeley and Lawrence Livermore Lab
            \begin{itemize}
                \item Developed reactor simulations
                \item Propagated uncertainties through them
                \item Helped design a hybrid fusion-fission reactor
            \end{itemize}

            \pause

            \item postdoc
            \begin{itemize}
                \item Developed reactor simulations
                \item Taught a course on nuclear reactor physics
                \item Helped design a thorium-fueled breed and burn reactor
            \end{itemize}
        \end{itemize}

    \end{frame}

\section{Nuclear Energy}

    \subsection{The state of the neutron}

        \begin{frame}{Nuclear energy is important right now}

            \begin{columns}[T]

                \begin{column}{0.5\textwidth}
                    \begin{figure}
                        \centering
                        \includegraphics{./img/sources.pdf}
                        \caption*{US elecriticity sources (2014)}
                    \end{figure}
                \end{column}

                \begin{column}{0.5\textwidth}
                    \begin{itemize}
                        \item Fossil Fuels generate 66\%
                        \pause
                        \item Nuclear generates 20\%
                        \pause
                        \item .. .and 60\% of carbon-free electricity
                        \pause
                        \item The most recent plant was built in 1996
                        \pause
                        \item ... its construction began in 1978
                        \pause
                        \item Renewables are ``other''
                    \end{itemize}
                \end{column}

            \end{columns}

        \end{frame}

        \begin{frame}{... and soon will be even more important!}

            \begin{itemize}

                \item Coal particulates cause 10 thousand \\ deaths annually in the US
                \pause
                \item One quarter of California air pollution is from China
                \pause
                \item Climate change is just getting ramped up
                \pause

                \vspace{2em}

                \item Global energy use will increase by a third by 2040
                \pause
                \item Natural gas and oil just got a lot cheaper
                \pause
                \item Coal is and always will be cheap

            \end{itemize}

        \end{frame}

        \begin{frame}{Nuclear reactors power 1800's era steam engines}{... as do coal, oil, and some natural gas}
            \begin{itemize}
                \item Nuclear chain reactions release heat
                \pause
                \item ... which boils water
                \pause
                \item ... which turns a turbine
                \pause
                \item ... which rotates a generator
                \pause
                \item ... which generates electricity!
                \pause
                \item This \href{https://www.youtube.com/v/VJfIbBDR3e8?start=0&end=165}{\beamergotobutton{Video}} explains it better than me
            \end{itemize}
        \end{frame}

    \subsection{We've come a long way since 1978}

        \begin{frame}{Sodium-cooled Fast reactors (SFR)}{... can consume nuclear waste and then recycle it and consuming it again}
            \begin{figure}
                \centering
                \includegraphics[width=1.0\textwidth]{./img/fastCycle.png}
                \caption*{}
            \end{figure}
        \end{frame}

        \begin{frame}{Traveling wave reactors (TWR)}{... (also known as Breed \& Burn reactors) can breed their own fuel \\ with zero reprocessing}
            \begin{figure}
                \centering
                \includegraphics[width=0.9\textwidth]{./img/candle.png}
                \caption*{}
            \end{figure}
        \end{frame}

        \begin{frame}{Fluoride-salt high-temperature reactors (FHR)}{... use coated particle fuel which cannot melt}
            \begin{figure}
                \centering
                \includegraphics[width=1.0\textwidth]{./img/fhrPebble.png}
                \caption*{}
            \end{figure}
        \end{frame}

        \begin{frame}{FHR's}{... also use fluoride salts instead of water \\ for compact operation at high-temperatures}
            \begin{figure}
                \centering
                \includegraphics[width=0.8\textwidth]{./img/fhrFlibe.png}
                \caption*{}
            \end{figure}
        \end{frame}

        \begin{frame}{FHR's}{... also use compact turbines \\ for high-efficiency energy conversion and load following}
            \begin{figure}
                \centering
                \includegraphics[width=0.9\textwidth]{./img/fhrPower.png}
                \caption*{}
            \end{figure}
        \end{frame}

        \begin{frame}{FHR's}{... can be passively cooled with ambient air \\ for extremely good safety}
            \begin{figure}
                \centering
                \includegraphics[width=0.9\textwidth]{./img/fhrBop.png}
                \caption*{}
            \end{figure}
        \end{frame}

        \begin{frame}{Hybrid fusion-fission reactors}{... still need some work}
            \begin{figure}
                \centering
                \includegraphics[width=0.9\textwidth]{./img/lifeFuel.png}
                \caption*{}
            \end{figure}
        \end{frame}

        \begin{frame}{Hybrid fusion-fission reactors}{.. are envisioned as a stepping-stone between fission and fusion energy}
            \begin{figure}
                \centering
                \includegraphics[width=1.0\textwidth]{./img/nifChamber.png}
                \caption*{}
            \end{figure}
        \end{frame}

        \begin{frame}{Hybrid fusion-fission reactors}{.. are awesome}
            \begin{figure}
                \centering
                \includegraphics[width=0.9\textwidth]{./img/lifeChamber.png}
                \caption*{}
            \end{figure}
        \end{frame}

\section{Simulations}

    \subsection{Solving the neutron transport equation}

        \begin{frame}{Accurate simulation of neutron fields requires their description 7-dimensional neutron phase space}
            \begin{figure}
                \centering
                \includegraphics[width=0.9\textwidth]{./img/phaseSpace.png}
            \end{figure}
        \end{frame}

        \begin{frame}{The neutron transport equation balances sources and sinks within this neutron phase space}{(time dependence is omitted for simplicity)}
            \begin{equation*}
                \begin{split}
                    \vec \Omega \cdot \vec \bigtriangledown \; \; \psi(\vec r, E, \vec \Omega) \\
                    + \sigma_{total}(\vec r, E) \; \; N(\vec r) \; \; \psi(\vec r, E, \vec\Omega) \\
                    = \int_0^\infty \! \! \! \! dE \int_{4\pi} \! \! \! \! d\vec\Omega \; \; \sigma_{scatter}(\vec r, E^\prime \rightarrow E, \vec \Omega^\prime \cdot \vec \Omega) \; \; N(\vec r) \; \; \psi(\vec r, E^\prime, \vec\Omega^\prime) \\
                    + \frac{\chi(E)}{4\pi} \int_0^\infty \! \! \! \! dE^{\prime\prime} \; \; \nu (E^{\prime\prime}) \; \; \sigma_{fission}(\vec r, E^{\prime\prime}) \; \; N(\vec r) \; \; \psi(\vec r, E^{\prime\prime}, \vec\Omega^{\prime\prime}) \\
                    + \mathcal{S}_{ext}(\vec r, E, \vec\Omega)
                \end{split}
            \end{equation*}
        \end{frame}

        \begin{frame}{Spatial distributions $(\vec r)$ can be complicated}
            \begin{figure}
                \centering
                \includegraphics[width=0.7\textwidth]{./img/spaceFlux1.png}
                \caption*{}
            \end{figure}
        \end{frame}

        \begin{frame}{Spatial distributions $(\vec r)$ can be complicated}
            \begin{figure}
                \centering
                \includegraphics[width=0.7\textwidth]{./img/spaceFlux2.png}
                \caption*{}
            \end{figure}
        \end{frame}

        \begin{frame}{Spatial distributions $(\vec r)$ can be complicated}
            \begin{figure}
                \centering
                \includegraphics[width=0.7\textwidth]{./img/spaceFlux3.png}
                \caption*{}
            \end{figure}
        \end{frame}

        \begin{frame}{Energy distributions $(E)$ can be complicated}
            \begin{figure}
                \centering
                \includegraphics[width=1.0\textwidth]{./img/energyXs.png}
                \caption*{}
            \end{figure}
        \end{frame}

        \begin{frame}{Energy distributions $(E)$ can be complicated}
            \begin{figure}
                \centering
                \includegraphics[width=1.0\textwidth]{./img/energyFlux.png}
                \caption*{}
            \end{figure}
        \end{frame}

        \begin{frame}{Simplification, Discretization, Monte Carlo}
        \end{frame}

        \begin{frame}{Coupled simulations}
        \end{frame}

    \subsection{Solving the Bateman equations}

        \begin{frame}{The Bateman equations describe the time-evolution of isotopes during decay and irradiation}{... it must be solved simultaneously with the NTE}
            \begin{equation*}
                \begin{split}
                    \frac{\partial N_i(\vec r, t)}{\partial t} \\
                    + \lambda_i \; \; N_i(\vec r, t) \\
                    + \int_0^\infty \! \! \! \! dE \; \; \sigma_{absorption,i} ( \vec r, E) \; \; \int_{4\pi} \! \! \! \! d\vec\Omega \; \; \psi(\vec r, E, \vec \Omega) \; \; N_i(\vec r, t) \\
                    = \sum_j \left[ b_{j \rightarrow i} \; \; \lambda_j \; \; N_j(\vec r, t) \right] \\
                    + \sum_k \left[ b_{k \rightarrow i} \int_0^\infty \! \! \! \! dE \; \; \sigma_{absorption,k} ( \vec r, E) \; \; \int_{4\pi} \! \! \! \! d\vec\Omega \; \; \psi(\vec r, E, \vec \Omega) \; \; N_k(\vec r, t) \right] \\
                \end{split}
            \end{equation*}
        \end{frame}

        \begin{frame}{Isotope depletion, breeding, and decay $(t)$ can be complicated}
            \begin{figure}
                \centering
                \includegraphics[width=0.7\textwidth]{./img/fuelCycle.png}
                \caption*{}
            \end{figure}
        \end{frame}

        \begin{frame}{Analytical, Matrix Exponential, CRAM}
        \end{frame}

\section*{Summary}

    \begin{frame}{Summary}
    \end{frame}

\end{document}
